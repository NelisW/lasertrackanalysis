% -*- TeX -*- -*- UK -*- -*- Soft -*-
\chapter{Scenario-Based Design}

\section{Definition}

\ac{SBD} considers a system in the context of a collection of comprehensive environments (called scenarios), expressed as a set of numbers, geometries or input conditions, that uniquely describes the environment, within which the system performs its functions.
Most \ac{SBD} scenarios represent and are defined in terms of commonly encountered and well understood user experiences or conditions, such as climatic conditions, \ac{3D} world and object descriptions, use cases, \ac{ConOps} outcomes, etc.
Seen from a missile development perspective, a scenario can be considered the design of a field test trial (e.g., a missile live firing) in terms of \ac{3D} geometry, climatic conditions, flight paths, military deployments, etc.
So in short a collection of scenarios can be seen as a set of well-considered field trial designs, to be used to prove system performance.

\ac{SBD} has the following key characteristics or objectives:
\begin{enumerate}

\item A user-centric design approach: determine the user's need in practical terms, and design to satisfy this need.
    
\item Demonstrate real-world performance under real-world conditions. Well-defined scenarios should encapsulate the system's the wider application scope. 

\item View the system from its application perspective: focus on what is important in the application domain.

\item Design the system from a holistic point of view: expressed as a collection of well-defined big-picture input conditions under which a well-defined outcome must emerge.

\end{enumerate}


The progression of interaction between an advanced system and its environment is most often not reducible to one or two simple numbers. For example, the detection distance of the missile is a simple number but may vary widely under different target or climatic conditions.  The only way such a complex system can be specified and tested is to `freeze' all the input conditions to a fixed single set and then determine the outcome for this specific set.  There must be a sufficient number of such frozen sets to cover the full scope of the system's intended operation. 

\section{}