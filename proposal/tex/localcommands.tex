% -*- TeX -*- -*- UK -*- -*- Soft -*-


% here we can keep commands etc. to no clutter the root file

\hyphenation{prod-uct}
\hyphenation{micro-bolo-meter}

\usepackage{blindtext}
\usepackage{multirow}
\usepackage{comment}
\usepackage{amsmath}
\usepackage{enumitem}


\usepackage{imakeidx}
\usepackage[indentunit=15pt,justific=raggedright, columnsep=10pt, font=normalsize]{idxlayout}
% to make index and hyperref work together, otherwise formatting suppresses hyperref
\newcommand{\BH}[1]{\hyperpage{#1}}
% https://tex.stackexchange.com/questions/575684/formatting-a-two-column-index-with-subitems
\makeatletter
\let\ori@idxitem\@idxitem
\def\@idxitem{\clear@penalties\ori@idxitem}
\def\clear@penalties{\subitem@count=3 }
\newcount\subitem@count
\def\subitem{%
  \advance\subitem@count -1
  \par
  \ifnum\subitem@count>0 \penalty10000 \fi
  \ori@idxitem\hspace*{\ila@subindent}%
}
\makeatother


\usepackage{amssymb}
\usepackage{pdfpages}
\usepackage{subcaption}

\newcommand{\umlder}{$\lhd\!-$} % left hand diamond, for uml derived class

\setcounter{section}{0}
\setcounter{secnumdepth}{6}


\definecolor{LightGrey}{rgb}{0.95,0.95,0.95}
\definecolor{light-gray}{gray}{0.95}
\definecolor{half-gray}{gray}{0.75}
\definecolor{LightRed}{rgb}{1.0,0.9,0.9}
\definecolor{brickred}{rgb}{0.8, 0.25, 0.33}
\definecolor{midnightblue}{rgb}{0.1, 0.1, 0.44}
\definecolor{pinegreen}{rgb}{0.0, 0.47, 0.44}
\definecolor{blizzardblue}{rgb}{0.67, 0.9, 0.93}
\definecolor{bondiblue}{rgb}{0.0, 0.58, 0.71}  
\definecolor{mygreen}{rgb}{0,0.6,0}
\definecolor{mygray}{rgb}{0.5,0.5,0.5}
\definecolor{mymauve}{rgb}{0.58,0,0.82}

\sisetup{per-mode=symbol}
\DeclareSIUnit\atm{atm}
\DeclareSIUnit\torr{torr}
\newcommand{\myrad}{\si{\watt\per(\steradian.\metre\squared)}} 
\newcommand{\myirrad}{\si{\watt\per(\metre\squared)}} 
\newcommand{\mymps}{\si{\meter\per\second}}
\newcommand{\mympssq}{\si{\meter\per\second\squared}}
\newcommand{\mydps}{\si{\degree\per\second}}
\newcommand{\mym}{\si{\metre}}
\newcommand{\mymm}{\si{\milli\metre}}
\newcommand{\mykm}{\si{\kilo\metre}}
\newcommand{\myum}{\si{\micro\metre}}
\newcommand{\myus}{\si{\micro\second}}
\newcommand{\myms}{\si{\milli\second}}
\newcommand{\myuw}{\si{\micro\watt}}
\newcommand{\myua}{\si{\micro\ampere}}
\newcommand{\mymv}{\si{\milli\volt}} 
\newcommand{\myuv}{\si{\micro\volt}} 
\newcommand{\myusr}{\si{\micro\steradian}}
\newcommand{\onetwo}{1\nobreak--\nobreak2~\si{\micro\metre}}
\newcommand{\onethree}{1\nobreak--\nobreak3~\si{\micro\metre}}
\newcommand{\pbsband}{1.5\nobreak--\nobreak2.5~\si{\micro\metre}}
\newcommand{\threefive}{3\nobreak--\nobreak5~\si{\micro\metre}}
\newcommand{\eighttwelve}{8\nobreak--\nobreak12~\si{\micro\metre}}
\newcommand{\mydegree}{\si{\degree}}
\newcommand{\degC}[1]{\SI{#1}{\celsius}}
\newcommand{\mymthreedb}{$-$3~dB}
\newcommand{\mymaththreedb}{-3\;{\rm dB}}
\newcommand{\deestarunits}{\si{\centi\meter\sqrt{\hertz}\per\watt}}

% this command allows the use of strings with underscore in text, captions and sections
% https://tex.stackexchange.com/questions/386953/url-in-section-title
% works with and requires hyperref, if not using hyperref, just use \url{string}
% the name comes from replace underscore
\newcommand{\rusc}[1]{\texorpdfstring{\nolinkurl{#1}}{#1}}

% upright derivative and similar
%https://tex.stackexchange.com/questions/84302/what-is-the-difference-of-mathop-operatorname-and-declaremathoperator
%https://tex.stackexchange.com/questions/1050/whats-the-difference-between-newcommand-and-newcommand
% see also https://ctan.org/pkg/physics?lang=en 
\newcommand{\diff}[1]{\operatorname{d\!}{}^{#1}}
\newcommand{\euler}{\operatorname{e\!}{}}
\newcommand{\imag}{\operatorname{j\!}{}}


\newcommand{\immediateneed}{\textbf{Immediate Need: }}
\newcommand{\futureneed}{\textbf{Future Need: }}

\newcommand{\scesys}{\ac{SceSyS}}
\newlist{legalout}{enumerate}{1}
\setlist[legalout]{label*=\thesubsection.\arabic*}
\newlist{legal}{enumerate}{10}
\setlist[legal]{label*=.\arabic*}



\newcommand{\colheightrule}{\rule[-2mm]{0mm}{6.5mm}}
\newcommand{\res}{\marginpar{$\ast$}}
\newcommand{\spec}[1]{\fcolorbox{half-gray}{light-gray}{#1}}

\newcommand{\Optional}{\textbf{Optional.\ }}
\newcommand{\Desirable}{\textbf{Desirable.\ }}
\newcommand{\Mandatory}{\textbf{Mandatory.\ }}


%******************  COMMENTS  **************
% in the body text:
%      text is marked with \begin{xxxx}...\end{xxxx} blocks.
% in the doc header you can instruct
%      \includecomment{xxxx} to include  the marked block
%  or  \specialcomment{xxxx} to include  the marked block
%      \excludecomment{xxxx} to exclude the marked block
% It is important that the \begin{xxxx} and \end{xxxx} commands appear on a
% separate line without whitespace characters before or after.
%


\newcommand{\getColorSpec}[3][\getColorSpecTemp]{%
  \extractcolorspec{#3}\getColorSpecTemp
  \expandafter\convertcolorspec\getColorSpecTemp{#2}#1}
\makeatletter
\newcommand\meaningbody[1]{%
  {\ttfamily
    \expandafter\strip@prefix\meaning#1}%
}
\makeatother
\newcommand{\getHTMLcolour}{\getColorSpec[\HTMLcolor]{HTML}{.}\texttt{\meaningbody\HTMLcolor}}


\specialcomment{note}{\begingroup\vspace{1mm}\color{bondiblue}\sffamily\small \hrule\textbf{\scshape \underline{Notes:}}%
\begin{cenumerate}}{\end{cenumerate}
\vspace{1mm}\hrule\vspace{3mm}\endgroup}

\specialcomment{ati}{\begingroup\vspace{1mm}\color{brown}\sffamily\small \hrule\textbf{\scshape \underline{Acceptance Test Instruction:}}\\Required actions to show compliance with requirement for this clause and/or sub-clauses:%
\begin{cenumerate}}{\end{cenumerate}
\vspace{1mm}\hrule\vspace{3mm}\endgroup}

\specialcomment{compliance}{\begingroup\vspace{1mm}\color{blue}\sffamily\small \hrule\textbf{\scshape \underline{Acceptance Test Compliance:}}\\Compliance/acceptance is demonstrated for this clause and/or sub-clauses as follows:%
\begin{cenumerate}}{\end{cenumerate}
\vspace{1mm}\hrule\vspace{3mm}\endgroup}

\specialcomment{todo}{\begingroup\vspace{1mm}\color{brickred}\sffamily\small \hrule\textbf{\scshape \underline{To Do:}}%
\begin{cenumerate}}{\end{cenumerate}
\vspace{1mm}\hrule\vspace{3mm}\endgroup}

\specialcomment{completed}{\begingroup\vspace{1mm}\color{pinegreen}\sffamily\small \hrule\textbf{\scshape \underline{Completed:}}%
\begin{cenumerate}}{\end{cenumerate}
\vspace{1mm}\hrule\vspace{3mm}\endgroup}

\specialcomment{halconprivate}{\begingroup\vspace{1mm}\color{red}\sffamily\small \hrule\textbf{\scshape \underline{Halcon Private:}}
\begin{cenumerate}}{\end{cenumerate}
\vspace{1mm}\hrule\vspace{3mm}\endgroup}




%\setlist{nolistsep}
\newlist{cenumerate}{enumerate}{2}
\setlist[cenumerate]{topsep=0pt,partopsep=0pt,parsep=0pt,itemsep=0pt,labelsep=1ex,label=\arabic*)}

\newcommand{\marginnote}[1]{ \marginpar{{\scriptsize #1}}}
\newcommand{\CJWpar}[2]{
\parbox{#1}{\rule{0cm}{4mm}#2\rule[-2mm]{0cm}{4mm}}
}
   
\newlength{\textwidttitlebox}
\setlength{\textwidttitlebox}{90mm}



\newcommand{\modtran}{\textsc{Modtran}}

